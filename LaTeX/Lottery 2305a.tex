\documentclass[letterpaper,12pt]{article}
% packages used
    \usepackage{natbib}
	\usepackage{threeparttable}
	\usepackage[format=hang,font=normalsize,labelfont=bf]{caption}
	\usepackage{amsmath}
	\usepackage{amssymb}
	\usepackage{amsthm}
	\usepackage{caption}
	\usepackage{subcaption}
	\usepackage{setspace}
	\usepackage{float,color}
	\usepackage[pdftex]{graphicx}
	\usepackage{hyperref}
	\usepackage{multirow}
	\usepackage{float,graphicx,color}
	\usepackage{graphics}
    \usepackage{placeins}
    \usepackage{authblk}
    \usepackage{tikz}

% other setup
	\hypersetup{colorlinks, linkcolor=red, urlcolor=blue, citecolor=red, hypertexnames=false}
	\graphicspath{{./figures/}}
	\DeclareMathOperator*{\Max}{Max}
	\bibliographystyle{aer}
	\numberwithin{equation}{section}
	\numberwithin{figure}{section}
	\numberwithin{table}{section}
	\newcommand\ve{\varepsilon}


\begin{document}

\begin{titlepage}
	\title{Playing the Lottery: Even When It's a Good Deal, It's Not}

	\author[1]{Kerk L. Phillips}

	\affil[1]{\footnotesize US Congressional Budget Office, Washington, DC, USA}


	\date{May 24, 2023\\
	\small{version 2023.05.a}}

	
	\maketitle

	\vspace{-0.3in}
	\begin{abstract}
	\small{
	This paper shows that for almost any utility function and distribution of expected income, playing the lottery decreases expected utility even in cases where the expected payoff is greater than the price of a lottery ticket.

	\vspace{0.1in}

	\textit{keywords:} lottery, risk aversion, welfare analysis, numerical analysis

	\vspace{0.1in}

	\textit{JEL classifications: D0, D6, H8} }
	\end{abstract}

	\centering
	IN PROGRESS

	\thispagestyle{empty}
\end{titlepage}

\begin{spacing}{1.5}

\section{Introduction} \label{sec_intro}

	As a general rule the expected value of a lottery ticket is less than the price.  This is because the lottery is a money-generation process for the government that runs it and revenue inflows must, on average, exceed the payment outflows.  However, with some lotteries on some occasions, when there is no jackpot winner and the jackpot rolls over for several consecutive weeks, the expected payoff can exceed the price of the ticket.  For example, the odds of matching all the numbers for the weekly Powerball lottery are approximately 292 million to one, while the odds for the MegaMillions lottery are 302 million to one.  The price for both lottery tickets is two dollars.  The average payout for the highest 5 jackpots from these two lotteries is 1.57 billion dollars.  This gives an approximate expected return of \$5.29 on a two-dollar lottery ticket, or an expected return of 164\%.  Even accounting for the fact that the present values of jackpots are substantially smaller than the numbers reported, this is still a sizable expected return on an investment.

	In these cases, is it ever worthwhile to purchase a lottery ticket?  On one hand, the odds of winning are overwhelmingly small, and the \$2 cost of the ticket is a virtually guaranteed loss.  On the other hand, in the that one-in-300 million case the payout is worth more than a billion dollars.  Which of these two dominates in expected utility terms?  Not surprisingly, the answer depends on the purchaser's risk aversion.  A risk-neutral purchaser would clearly be better off purchasing the ticket.  However, most people are risk averse to some degree or another.  In this paper I show that for a wide ranges of parameterized utility functions, there is a net loss in expected utility when purchasing a lottery ticket.  This result is quite intuitive as it is hard to think of an asset purchase that introduced more risk that a lottery ticket.  By comparison, the odds of being killed by lightning over the course of a lifetime in the U.S. are one in 153,000, or a little over 11 million to one in a given year.  In order for a lottery ticket to generate gains in expected utility, risk aversion must be unreasonably small, or the the odds of winning must be unreasonably high.

\section{Numerical Analysis} \label{sec_numer}

	In this section we calculate the expected welfare loss or gain from playing the lottery.  We assume that a lottery player already has a stochastic endowment of non-lottery income, denoted $x$ which will all be spent on consumption or the purchase of a single lottery ticket.  Utility from consumption comes from one of the five functional forms in Section \ref{sec_util}.  The distribution of non-lottery income is described by one of the probability density functions from Section \ref{sec_dist}.  The parameters of the lottery are laid out in Section \ref{sec_lottery}.

	If the individual chooses not to play the lottery, their consumption ($c$) is exactly equal to their stochastic non-lottery income, so $c = x$.  If they choose to play the lottery then with probability $\tfrac{1}{\omega}$ their consumption will be their non-lottery income plus the lottery payoff ($p$) minus the cost of the lottery ticket ($t$), giving $c = x + p - t$.  However, if they lose the lottery, their consumption is non-lottery income minus the cost of the ticket, $c = x - t$

	Disposable income is discretized over a grid with $n$ equally spaced points between $x_{min}$ and $x_{max}$.  The probabilities are generated using the density functions below and are scaled appropriately so that they sum to 1 over the $n$ points in the grid.


	\subsection{Utility Functions} \label{sec_util}

		Constant Relative Risk Aversion (CRRA)
		\begin{equation}
			u(c) = \frac{c^{1-\gamma}-1}{1-\gamma}
		\end{equation}

		Stone-Geary
		\begin{equation}
			u(c) = \frac{(c-\underline{c})^{1-\gamma}-1}{1-\gamma}
		\end{equation}

		Exponential
		\begin{equation}
			u(c) = 1-e^{-a c}
		\end{equation}

		Hyperbolic Absolute Risk Aversion (HARA)
		\begin{equation}
			u(c) = \frac{1-\gamma}{\gamma} \left( \frac{a c}{1-\gamma} + b \right)^\gamma
		\end{equation}

		Logarithmic
		\begin{equation}
			u(c) = \ln c
		\end{equation}

	\subsection{Non-Lottery Income Distributions} \label{sec_dist}

		Beta Probability Density Function
		\begin{equation}
			f(x) = \frac{\left( \frac{x-x_{min}}{x_{max}-x_{min}} \right)^{a-1} \left(1 - \frac{x-x_{min}}{x_{max}-x_{min}} \right )^{b-1}} {B(a,b)}
		\end{equation}
		where $B(a,b)$ is the beta function.  In this paper $a=2, b=2.2$.

		Normal Probability Density Function
		\begin{equation}
			f(x) = \frac{1}{\sigma \sqrt{2 \pi}} e^{-\frac{1}{2} \left(\frac{x-\mu}{\sigma} \right)^2}
		\end{equation}
		where $\mu = \frac{x_{max}+x_{min}}{2}, \sigma = \frac{x_{max}-x_{min}}{3}$.

		Uniform Probability Density Function
		\begin{equation}
			f(x) = q = \frac{1}{n}
		\end{equation}	

	\subsection{Lottery Setup} \label{sec_lottery}

		A lottery ticket costs 2 units and pays 1.5 billion units with odds of 1:300,000,000.  This gives a gross expected return of 250 percent.  The net change in expected utility from buying a lottery ticket is given by the following formula.
		\begin{equation}
			\Delta E\{u\} = \frac{\omega - 1}{\omega} E\{u'(x - t)| \ell = 0 \} + \frac{1}{\omega} E\{u'(x - t + p)| \ell = 1 \}
		\end{equation}
		where $E\{.\}$ is the expectations operator, $u'(.)$ is the marginal utility function, and $\ell$ is an indicator variable equal to 1 if the winning ticket is purchased (with probability $\tfrac{1}{\omega}$) and 0 otherwise.

\section{Results} \label{sec_results}

	I run an initial set of calculations using a grid for $x$ with $n=101$, $x_{min} = 20,000$ and $x_{max} = 150,000$.  This corresponds to an individual with an expected non-lottery income between $70,000$ and $85,000$, but high variance in that income.  The results of these calculations are shown in Table \ref{tab_lottery1}. In every case expected consumption rises by three dollars despite a drop in two dollars due to the cost of a lottery ticket.  However, expected utility falls in every case as well.

	\begin{table}[ht] 
		\caption{Changes in Utility with High Variance}
		\label{tab_lottery1}
		\centering
		\begin{tabular}{|ll|lll|}
		    \hline
			$u(c)$ & $f(x)$ & High Variance & Low Income & High Income \\
			\hline
			CRRA & beta & -6.19E-10 & -3.54E-09 & -9.65E-11 \\
			CRRA & normal & -8.43E-10 & -4.30E-09 & -1.02E-10 \\
			CRRA & uniform & -6.92E-10 & -3.35E-09 & -9.53E-11 \\
			\hline
			S-G & beta & -1.09E-09 & -1.08E-08 & -1.11E-10 \\
			S-G & normal & -1.84E-09 & -1.52E-08 & -1.18E-10 \\
			S-G & uniform & -1.53E-09 & -1.01E-08 & -1.10E-10 \\
			\hline
			exponential & beta & -2.46E-15 & -3.27E-13 & -2.51E-65 \\
			exponential & normal & -6.07E-14 & -1.45E-12 & -2.65E-64 \\
			exponential & uniform & -5.68E-14 & -4.34E-13 & -3.33E-65 \\
			\hline
			HARA & beta & -5.45E-03 & -8.97E-03 & -3.63E-03 \\
			HARA & normal & -5.71E-03 & -9.41E-03 & -3.68E-03 \\
			HARA & uniform & -5.27E-03 & -8.82E-03 & -3.62E-03 \\
			\hline
			log & beta & -3.23E-05 & -8.39E-05 & -1.39E-05 \\
			log & normal & -3.62E-05 & -9.25E-05 & -1.43E-05 \\
			log & uniform & -3.14E-05 & -8.13E-05 & -1.38E-05 \\
			\hline
		\end{tabular}
	\end{table}

	\begin{table}[ht] 
		\caption{Values of $\gamma$ That Make the Change in Utility Equal Zero}
		\label{tab_lottery2}
		\centering
		\begin{tabular}{|l|ll|}
		    \hline
			$f(x)$ & CRRA & S-G \\
			\hline
			beta & 0.1022 & 0.1001 \\
			normal & 0.1013 & 0.0990 \\
			uniform & 0.1033 & 0.1011 \\
			\hline
		\end{tabular}
	\end{table}

	\begin{table}[ht] 
		\caption{Odds That Make the Change in Utility Equal Zero}
		\label{tab_lottery3}
		\centering
		\begin{tabular}{|l|ll|ll|}
		    \hline
			$f(x)$ & \multicolumn{2}{c}{CRRA} & \multicolumn{2}{c}{S-G} \\
			& odds & return & odds & return \\
			\hline
			beta & 1:26000 & 288462\% & 1:18600 & 403226\% \\
			normal & 1:21500 & 348837\% & 1:13200 & 568182\% \\
			uniform & 1:22700 & 330396\% & 1:13600 & 551471\% \\
			\hline
		\end{tabular}
	\end{table}


\FloatBarrier

\section{Conclusions} \label{sec_concl}
	

%\newpage
%\nocite{*}
%\bibliography{Lottery}

\end{spacing}

\end{document}