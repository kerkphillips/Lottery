\documentclass[letterpaper,12pt]{article}
% packages used
    \usepackage{natbib}
	\usepackage{threeparttable}
	\usepackage[format=hang,font=normalsize,labelfont=bf]{caption}
	\usepackage{amsmath}
	\usepackage{amssymb}
	\usepackage{amsthm}
	\usepackage{caption}
	\usepackage{subcaption}
	\usepackage{setspace}
	\usepackage{float,color}
	\usepackage[pdftex]{graphicx}
	\usepackage{hyperref}
	\usepackage{multirow}
	\usepackage{float,graphicx,color}
	\usepackage{graphics}
    \usepackage{placeins}
    \usepackage{authblk}
    \usepackage{tikz}

% other setup
	\hypersetup{colorlinks, linkcolor=red, urlcolor=blue, citecolor=red, hypertexnames=false}
	\graphicspath{{./figures/}}
	\DeclareMathOperator*{\Max}{Max}
	\bibliographystyle{aer}
	\numberwithin{equation}{section}
	\numberwithin{figure}{section}
	\numberwithin{table}{section}
	\newcommand\ve{\varepsilon}


\begin{document}

\begin{titlepage}
	\title{Playing the Lottery: Even When It's a Good Deal, It's Not}

	\author[1]{Kerk L. Phillips}

	\affil[1]{\footnotesize US Congressional Budget Office, Washington, DC, USA}


	\date{May 24, 2023\\
	\small{version 2023.05.a}}

	
	\maketitle

	\vspace{-0.3in}
	\begin{abstract}
	\small{
	This paper shows that for almost any utility function and distribution of expected income, playing the lottery decreases expected utility even in cases where the expected payoff is greater than the price of a lottery ticket.

	\vspace{0.1in}

	\textit{keywords:} lottery, risk aversion, welfare analysis, numerical analysis

	\vspace{0.1in}

	\textit{JEL classifications: D0, D6, H8} }
	\end{abstract}

	\centering
	IN PROGRESS

	\thispagestyle{empty}
\end{titlepage}

\begin{spacing}{1.5}

\section{Introduction} \label{sec_intro}

	As a general rule the expected value of a lottery ticket is less than the price.  This is because the lottery is a money-generation process for the government that runs it and revenue inflows must, on average, exceed the payment outflows.  However, with some lotteries on some occasions, when there is no jackpot winner and the jackpot rolls over for several consecutive weeks, the expected payoff can exceed the price of the ticket.  For example, the odds of matching all the numbers for the weekly Powerball lottery are approximately 292 million to one, while the odds for the MegaMillions lottery are 302 million to one.  The price for both lottery tickets is two dollars.  The average payout for the highest 5 jackpots from these two lotteries is 1.57 billion dollars.  This gives an approximate expected return of \$5.29 on a two-dollar lottery ticket, or an expected return of 164\%.  Even accounting for the fact that the present values of jackpots are substantially smaller than the numbers reported, this is still a sizable expected return on an investment.

	In these cases, is it ever worthwhile to purchase a lottery ticket?  On one hand, the odds of winning are overwhelmingly small, and the \$2 cost of the ticket is a virtually guaranteed loss.  On the other hand, in the that one-in-300 million case the payout is worth more than a billion dollars.  Which of these two dominates in expected utility terms?  Not surprisingly, the answer depends on the purchaser's risk aversion.  A risk-neutral purchaser would clearly be better off purchasing the ticket.  However, most people are risk averse to some degree or another.  In this paper I show that for a wide ranges of parameterized utility functions, there is a net loss in expected utility when purchasing a lottery ticket.  This result is quite intuitive as it is hard to think of an asset purchase that introduced more risk that a lottery ticket.  By comparison, the odds of being killed by lightning over the course of a lifetime in the U.S. are one in 153,000, or a little over 11 million to one in a given year.  In order for a lottery ticket to generate gains in expected utility, risk aversion must be unreasonably small, or the the odds of winning must be unreasonably high.

\section{Numerical Analysis} \label{sec_numer}


\section{Conclusions} \label{sec_concl}
	

%\newpage
%\nocite{*}
%\bibliography{Lottery}

\end{spacing}

\end{document}